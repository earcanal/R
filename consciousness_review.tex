\documentclass[a4paper]{article}
\usepackage[grumpy]{gitinfo2}

\begin{document}

\title{Consciousness: an introduction}
\date{Version:\gitRels{}, \gitAuthorDate{} }
\author{paul.sharpe@plymouth.ac.uk}

\maketitle

This up-to-date introduction to consciousness studies is aimed principally at
psychology students and teachers. The book consists of eighteen chapters,
divided into six sections, and draws on a wide range of empirical and
theoretical research. Each chapter introduces a new topic and includes key
concept boxes, author profiles, and extensive cross-references within and
between sections. Individual and group activities help students to develop a
deeper exploration of the academic material, and their own consciousness. Each
chapter ends with readings that offer multiple perspectives to support seminar
discussions. There is a comprehensive index, eighty-five pages of references,
and a companion website which includes updates, instructor resources,
self-assessment materials, and a reference list with links to cited papers.

Section one begins with philosophical explanations of how mind and body might
interact, without having to explain away a ``ghost in the machine'', or mental
homunculus. Philosophical extremes of materialism and idealism are contrasted
with dualist, monist, epiphenomenalist, and panpsychist accounts of the
mind-body problem. The chapter then takes a historical look at psychological
approaches to studying consciousness. Behaviourism dominated psychology in the
mid-twentieth century, shutting down earlier attempts to explain consciousness
using introspective methods. The field was revived in the late twentieth
century, and contemporary approaches draw on psychology, neuroscience,
phenomenology and embodied cognition. These methods have offered solutions to
the (relatively!) easy problems of consciousness, but are challenged by the
harder problem of why a subjective, first person perspective exists at all.
This interplay between psychology and philosophy is used throughout the book
to show solutions to the easy and hard problems of consciousness.

Chapter two takes a closer look at the hard problem of consciousness.
Contrasing philosophical positions which claim that subjective experience is
either real, or illusory, are outlined using thought experiments. At the
extremes, some thinkers consider the hard problem is insoluble, or simply
doesn’t exist. Others have addressed the hard problem head-on or tried to
decompose the hard problem, making it more tractable. An alternative approach
has been to address the easy problems of consciousness, in the hope that these
may have the side effect of solving, or at least softening the hard problem.

The psychology of visual illusions exemplify some of these problems.
Experiments demonstrate that the visual system ``fills in'' perceptual gaps,
and is susceptible to errors when scenes change, or when there are lapses in
attention. These findings challenge descriptions of visual experience as a
stream of conscious mental images, represented in the brain. The science of
visual illusions has been used to support materialist, phenomenological and
illusionist accounts of consciousness.

Section two focuses the brain, beginning with an overview of its structure,
function and the neuroscientific methods. These methods have been central to
theories which explain consciousness in terms of correlations between neural
activity and visual experience. Consciousness studies have been informed by
anaesthesia, unconscious states associated with some clinical conditions, and
the neural correlates of pain. Embodied cognition and the nature of the self
emerge as perspectives which may have most to say about puzzling phenomena
including the experience of pain in ``phantom limbs''.

Chapter five contrasts conscious with unconscious experience. Many theories
explain psychological and neuroscientific evidence in this area using a
theatre metaphor. For example, global workspace theories describe degrees of
conscious experience as like varying levels of illumination from a ``mental
spotlight''. This resonates with our intuitive sense of conscious contents,
which converge in time and space, but risks introducing a mental homunculus
who views whatever appears under the spotlight. Multiple drafts theory avoids
having to explain an infinite regress of mental homunculi, by arguing that
consciousness creates the \textit{illusion} of a mental theatre. Other
theories avoid the same paradox by focussing exclusively on brain activity, or
drawing on quantum physics.

The apparent incongruency between complex, distributed neural activity and
unified conscious experiences is discussed in chapter six. Psychological
theories explain how visual features associated with distinct brain regions,
such as lines and colours, are bound together to form the visual objects that
we perceive, and can also account for multi-sensory feature binding. One
suggestion is that these unified experiences are the peaks in a hierarchy of
micro-consciousnesses. Other theories claim that embodiment is essential for 
the unification of experience, or that our unitary experience is an illusion.
Theories have been informed by extremes of both conscious unity, such as
synaesthesia, and disunity in patients with split brains, amnesia, and
hemispheric neglect. The chapter ends by exploring how conscious experience
comes to be unified in time.

Section three explores the role played by consciousness when bodies interact
with the world. Attention is closely linked with conscious awareness, and both
voluntarily and involuntarily visual attention are associated with distinct
patterns of brain activation. Scientific theories often describe attention in
terms of a resource, or neural activation which mediates conscious awareness.
Philosophers have aimed to unify the functional and phenomenal aspects of
attention. These scientific and philosophical foundations are used to explore
whether consciousness depends on attention or vice versa. Are attention and
consciousness simply correlated, or an epiphenomenon of more global
processing? Is attention even a useful category for understanding
consciousness? Attention plays a central role in meditation, and the chapter
ends by describing the psychological and neural effects of meditation
training.

The focus then shifts from attention to conscious and unconscious perception
and action. Social and emotional responses have been useful for measuring the
boundaries between conscious and unconscious perception. A categorisation of
actions based on degrees of consciousness is introduced, and functionalism is
described as a philosophical position which is central to differing opinions
as to whether consciousness can cause action. Visual perception seems to
require consciousness, whereas actions often occur without conscious
awareness. This dissociation is associated with distinct neural pathways.
These points are drawn together in a discussion about blindsight, a condition
where humans appear to be able to see without conscious awareness. The chapter
ends by looking at conscious and unconscious processing in relation to
intuition and creativity.

The final chapter in section three explores consciousness and free will.
First, the neuroanatomy of willed action is introduced. Evidence suggests that
conscious experience requires at least half a second of sensory stimulation. A
consequence of this is that subjective experience is projected backwards in
time. Other experiments appear to show that consciousness \textit{follows},
rather than precedes, the initiation of willed action. These findings have
created intense philosophical debate regarding the relationship between
consciousness and free will. The chapter ends by exploring the consequences of
evidence which suggest that a belief in agency may be an illusory association
between unconscious thought and action.

Section four examines the evolution of consciousness in humans and non-human
animals. It begins by considering whether consciousness evolves according to
evolutionary principles which are independent of gene propagation understood
to drive biological evolution. Next, brain structure and behaviour are
discussed as criteria for different views on the degree and types of
consciousness present in the animal kingdom. Evidence is evaluated regarding
the extent to which human and non-human animals have self awareness, and can
infer the existence of other minds. The ability to imitate and use language
are discussed as possible thresholds indicating different orders of
consciousness.

Chapter eleven uses the perspective of evolutionary psychology to examine
whether consciousness evolved as function distinct from other aspects of
cognition. Thought experiments are used to explore whether, why, when, and how
consciousness might have evolved. The extent to which consciousness evolved
due to biological advantages is discussed, along with the social advantages
that might naturally emerge from self-reflective consciousness. Phenomenal
consciousness, real or illusory, may also have evolved without requiring it to
have an independent function. The chapter ends by exploring how consciousness
might evolve in neural circuits and cultural memes.

The section ends by considering the mind as an evolved machine, and the
evolution of machine consciousness. Similarities between minds and machines
have made computing an important method for  understanding and attempting
create artificial consciousness. An outstanding question is whether these
questions can be answered using computers running artificial neural networks,
or whether embodiment in robots is necessary. A thought experiment is
introduced to distinguish machine intelligence from machine consciousness, and
assess arguments for whether the latter is possible, or has already happened!
Approaches to creating artificial consciousness have included embedding a
component essential to consciousness into machines, evolving languages which
include a self concept, and building social robots.

Section 5 examines the ``borderlands'' of altered states of consciousness
(ASCs), imagination and dreams. Challenges in this area include choosing norms
against which to define ``altered'', and distinguishing consciousness from
other aspects of cognition. No single classification scheme captures the
wide-ranging, often idiosyncratic phenomenology of ASCs. Altered states of
consciousness induced by psychoactive drugs, meditation, hypnosis and mental
ill health are described. The extent to which ASC phenomenology correlates
with physiology, neural activity or representational states is discussed in
relation to the mind-body problem.

Although humans can effortlessly discriminate between reality and imagination,
false perceptions and distortions to autobiographical memories are also
common. Extensive coverage is given to external and internal influences which
can generate hallucinations in various aspects of sensory experience.
Predictive processing emerges as a theory especially able to account for
hallucinations. There is little evidence for extra-sensory perception (ESP),
or explanations of ESP in terms of consciousness. The chapter ends with
accounts of borderline states between imagination and reality in children and
adults.

The last borderland to be explored are states of consciousness associated with
sleep and wakefulness. Some theories try to explain mappings between the
physiology and phenomenology of ordinary dreaming. Others question whether
dreams are experienced consciously during sleep, or composed unconsciously and
only experienced at waking. Borderline states of hypnagogia, false awakenings
and sleep paralysis are covered, and the chapter ends by reviewing empirical
research and theories of lucid dreaming, out of body, and near-death
experiences. The implications of these experiences are evaluated in relation
to philosophical positions on consciousness and the sense of self.

The final section looks at the self, and interactions between selves. Theories
which consider the self to be real and continuous, are contrasted with those
that claim that the sense of a continuous self is an illusion. These theories
are evaluated using thought experiments in relation to cases of multiple
personality, the idea that thoughts create the illusion of a thinker, and
neuroscientific models of normal and abnormal selves. Other theories consider
the self to be a high level, recursive mental representation, or an
aggregation of lesser, minimal selves. The self might also emerge through
embodied interaction with others, or be constructed through language. The
chapter ends by considering how human enhancement using technology, and the
Internet affect the self.

Next, contrasting views on the roles of first-person and second-person methods
in the scientific study of consciousness are discussed. A key first-person
method is phenomenology. This is described alongside neurophenomenology, which
combines phenomenology with third-person neuroscientific methods.
Neurofeedback and second-person methods are approaches which may bridge the
gap between first-person and third-person methods. The chapter ends with a
discussion of how heterophenomenology, a method of studying another person’s
phenomenology, might be superior to purely first-person methods.

The final chapter asks whether changes to consciousness can help to bridge the
mysterious gap between mind and body. The classical account of this process is
the Buddha’s awakening, or enlightenment. Buddhism is compared and contrasted
with science and transformation through psychotherapy. Cases of spontaneous
awakening outside of any spiritual framework are described, and the
possibility of establishing the neural correlates of enlightenment is
discussed. The book ends as it began, with the mind-body problem. Perhaps
non-dual states of consciousness, in which there is no distinction between
self and experience, could provide a direct solution to the problem.

This book contains something for everyone interested in understanding more
about consciousness. It manages to remain accessible, without compromising its
coverage of technical, philosophical and experiential topics. The book has
clearly been road-tested as a core text for structuring lectures and seminars,
and is also suitable for self-study. The authors' emphasis on the
first-person, subjective dimension of consciousness provides a unique,
experiential balance to some challenging material. They punctuate the text
with literary quotes, biographies of key thinkers, and descriptions of their
own experiences, helping readers to reflect on what consciousness is, and what
it’s like to be conscious being. Now.

\end{document}