\documentclass[a4paper]{article}
\usepackage[grumpy]{gitinfo2}

\begin{document}

\title{Consciousness: an introduction}
\date{\gitAuthorDate{} Release\gitRels}
\author{paul.sharpe@plymouth.ac.uk}

\maketitle

This up-to-date introduction to consciousness studies is aimed principally at
psychology students and teachers. The book consists of eighteen chapters,
divided into six sections, and draws on a wide range of empirical and
theoretical research. Each chapter introduces a new topic and includes key
concept boxes, author profiles, and extensive cross-references within and
between sections. Individual and group activities help students to develop a
deeper exploration of the academic material, and their own consciousness.
Chapter readings offer multiple perspectives to support seminar discussions,
and deeper explorations of the topic. There is a comprehensive index and
eighty-five pages of references. A companion website includes updates,
instructor resources, self-assessment materials, and a reference list with
links to cited papers.

Section one begins by outlining philosophical explanations of how mind and
body could interact, without a ``ghost in the machine'' or mental homunculus,
which would require further explanation. Dualist, monist, epiphenomenalist,
and panpsychist accounts of the mind-body problem are presented as
alternatives to the extremes of materialism and idealism. The chapter then
takes a historical look at psychological approaches to studying consciousness.
Behaviourism dominated psychology in the mid-twentieth century, shutting down
early attempts to explain consciousness using introspective methods. The field
was revived in the late twentieth century, and contemporary approaches draw on
psychology, neuroscience, phenomenology and embodied cognition. These methods
have begun to penetrate the (relatively!) easy problems of consciousness, but
are challenged by the harder problem of why a subjective, first person
perspective exists at all. This interplay between psychology and philosophy
runs throughout the book’s coverage of the easy and hard problems of
consciousness.

Chapter two takes a closer look at the hard problem of consciousness. Thought
experiments are used to portray philosophical positions which claim that
subjective experience as either real, or illusory. At the extremes, some
thinkers consider the hard problem is insoluble, or simply doesn’t exist.
Others have addressed the hard problem head-on, and these explanations are
presented alongside initiatives to decompose the hard problem, making it more
tractable. An alternative approach has been to address the easy problems of
consciousness, in the hope that these may have the side effect of solving, or
at least softening the hard problem.

The psychology of visual illusions is used to challenge intuitions about the
nature of conscious, visual experience. They demonstrate that the visual
system ``fills in'' perceptual gaps, and is susceptible to errors when scenes
change or there are lapses in attention. These effects challenge explanations
of visual experience as a stream of conscious mental images, represented in
the brain. Philosophers have used this empirical evidence to support
materialist, phenomenological and illusionist accounts of consciousness.

Consciousness and the brain is the focus of Section two. Chapter four begins
with an overview of brain structure, function and associated neuroscientific
methods. These methods have been central to theories which explain
consciousness in terms of correlations between neural activity and visual
experience. Studies of anaesthesia, and unconscious states associated with
some clinical conditions have been informative. The chapter ends by examining
the neural correlates of pain. Embodied cognition and the nature of the self
emerge as perspectives which may have most to say about puzzling phenomena
including the experience of pain in ``phantom limbs''.

Global workspace theories, amongst others, explain psychological and
neuroscientific evidence using a theatre metaphor. The contents of experience
enter and exit conscious awareness, to the extent that they are illuminated by
a moving ``mental spotlight''. This resonates with our intuitive sense of
conscious experience converging in time and space, but risks introducing a
mental homunculus who views whatever appears under the spotlight. Multiple
drafts theory avoids having to explain an infinite regress of mental homunculi
by arguing that consciousness creates the \textit{illusion} of a mental
theatre. Other theories avoid the paradox by focussing exclusively on brain
activity, or drawing on quantum physics.

The apparent incongruency between complex, distributed neural activity and
unified conscious experiences is discussed in chapter six. Psychological
theories can explain how visual features associated with distinct brain
regions, such as lines and colours, are bound together to form the visual
objects that we perceive. Integrated information theory can account for,
amongst many other things, multi-sensory feature binding. One suggestion is
that these unified experience are the peaks in a hierarchy of
micro-consciousnesses. Other theories claim that embodiment is essential for
the unification of experience, or that the sense of consciousness as unitary
is an illusion. At the extremes of conscious unity and disunity, synaesthesia,
split brains, amnesia, and hemispheric neglect have informed these theories.
The chapter ends by exploring how conscious experience is unified in time.

Section three explores the role of consciousness in interactions between the
body and the world. Attention is closely linked with conscious awareness, and
both voluntarily and involuntarily visual attention are associated with
distinct patterns of brain activation. Scientific theories are presented which
describe attention in terms of a resource, or neural activation mediating
conscious awareness. Philosophical accounts have aimed to unify the functional
and phenomenal aspects of attention. These theoretical and philosophical
foundations are used to explore whether consciousness depends on attention or
vice versa. Are attention and consciousness simply correlated, an
epiphenomenon of more global processing, or is attention even a useful
category for understanding consciousness? The chapter ends by describing the
psychological and neural effects of meditation, a type of mental training in
which attention plays a central role.

Next, the focus shifts from attention to the boundary between conscious and
unconscious perception and action. Theories and methods for measuring
perceptual thresholds are outlined, and unconscious processing is shown to be
detectable in social and emotional responses. Actions are categorised based on
the degree of consciousness required, and functionalism is described in
relation to differing opinions as to whether consciousness can, or cannot
cause action. Neurological and behavioural evidence suggests that there are
distinct pathways that allow motor responses without consciousness, but
require consciousness for visual perception. This background provides a
foundation for debates about consciousness in relation to blindsight, a
condition where humans appear to be able to see without conscious awareness.
The chapter ends by looking at conscious and unconscious processing in
relation to intuition and creativity.

The final chapter in section three explores consciousness and free will.
First, the neuroanatomy of willed action is introduced. Evidence suggests that
half a second of sensory stimulation is required for an experience to become
conscious. A consequence of this is that subjective experience is projected
backwards in time. Other experiments appear to show that consciousness
follows, rather than precedes, the initiation of willed action. These findings
have created intense philosophical debate regarding the relationship between
consciousness and free. The chapter ends by describing the consequences of
experiments which suggest that free will, and the belief in agency may be
illusory associations between unconscious thought and action.

Section four examines the evolution of consciousness in humans and non-human
animals. It begins by considering whether consciousness evolves according to
criteria beyond the propagation of genes understood to drive biological
evolution. Next, brain structure and behaviour are discussed as criteria for
different views on the degree and types of consciousness present in the animal
kingdom. Evidence is evaluated regarding the extent to which human and
non-human animals have self awareness, and can infer the existence of other
minds. The ability to imitate and use language are discussed as possible
thresholds indicating different orders of consciousness.

Shifting to a psychological perspective, chapter eleven examines whether
consciousness evolved as function distinct from other aspects of cognition.
Thought experiments are used to explore whether, why, when, and how
consciousness might have evolved. The extent to which consciousness evolved
due to biological advantages is discussed, and the social advantages that
might naturally emerge from self-reflective consciousness. Other theories
suggest that phenomenal consciousness, real or illusory, could have evolved
without requiring it to have an independent function. The chapter ends by
exploring the possibility that consciousness may evolve according to
evolutionary principles, in neural circuits and cultural memes, independently
of genes.

The examination of consciousness and evolution ends by considering the mind as
an evolved machine, and the evolution of machine consciousness. Similarities
between minds and machines have put computing at the centre attempts to
understand and create consciousness. An outstanding question is whether
artificial neural networks running on computers are sufficient, or whether
embodiment in robots is necessary. The ``Chinese room'' thought experiment is
introduced to distinguish machine intelligence from machine consciousness, and
assess arguments for whether the latter is possible, or has already happened.
Ongoing attempts to create artificial consciousness include embedding a
component essential to consciousness into a machine, computers which evolve a
self concept through language, and social robots.

Section 5 examines the ``borderlands'' of altered states of consciousness
(ASCs), imagination and dreams. Selecting norms against which to define ASCs
has been challenging, as has distinguishing consciousness from other aspects
of cognition. No single classification scheme seems to capture the
wide-ranging, often idiosyncratic phenomenology of ASCs. Altered states of
consciousness induced by psychoactive drugs, meditation, hypnosis and mental
ill health are described. The extent to which ASC phenomenology correlates
with physiology, neural activity or representational states is discussed in
relation to the mind-body problem.

Although humans can effortlessly discriminate between reality and imagination,
false perceptions and distortions to autobiographical memories are also
common. Extensive coverage is given to external and internal influences which
can generate hallucinations in various aspects of sensory experience.
Predictive processing emerges as a theory especially able to account for
hallucinations. There is little evidence for extra-sensory perception, or
explanations of ESP in terms of consciousness. The chapter ends with accounts
of borderline states between imagination and reality in children and adults.

The exploration of the borderlands ends by considering sleep and wakefulness
in relation to the mind-body problem. Some theories try to explain mappings
between the physiology and phenomenology of ordinary dreaming. Others question
whether dreams are experienced consciously during sleep, or composed
unconsciously and only experienced at waking. Borderline states of hypnagogia,
false awakenings and sleep paralysis are covered, and the chapter ends by
reviewing empirical research and theories of lucid dreaming, out of body, and
near-death experiences. The implications of these experiences are evaluated in
relation to philosophical positions on consciousness and the sense of self.

The final section looks at the self, and interactions between selves. Ego
theories, which consider the self to be real and continuous, are contrasted
with bundle theories, which propose that the sense of a continuous self is an
illusion. These theories are evaluated using thought experiments in relation
to cases of multiple personality, the idea that thoughts create the illusion
of a thinker, and neuroscientific models of normal and abnormal selves. Other
theories consider the self to be a high level, recursive mental
representation, or an aggregation of lesser, minimal selves. The self might
also emerge through embodied interaction with others, or be constructed
through language. The chapter ends by considering how human enhancement using
technology, and the Internet affect the self.

Next, contrasting views on the roles of first-person and second-person methods
in the scientific study of consciousness are discussed. A key first-person
method is phenomenology. This is described alongside neurophenomenology, which
combines phenomenology with third-person neuroscientific methods.
Neurofeedback and second-person methods are discussed as approaches which may
bridge the gap between first-person and third-person methods. The chapter ends
with a discussion of how heterophenomenology, a method of studying another
person’s phenomenology, might be superior to purely first-person methods.

The final chapter asks whether changes to consciousness can help to bridge the
mysterious gap between mind and body. The classical account of this process is
the Buddha’s awakening, or enlightenment. Science and Buddhism have
established a dialogue, which is evidence that they are compatible to some
extent. Buddhism is compared and contrasted with transformation through
psychotherapy, and cases of spontaneous awakening outside of any spiritual
framework are described. The possibility of establishing the neural correlates
of enlightenment is discussed. The book ends as it began, with the mind-body
problem. Perhaps non-dual states, in which there is no distinction between
self and experience, could provide a solution.

This book was a pleasure to read, and contains something for everyone
interested in understanding more about consciousness. It manages to remain
accessible while tackling challenging technical, philosophical and
experiential aspects, and has clearly been road-tested as a text suitable for
lectures, seminars or self-study. The authors' emphasis of the first-person,
subjective dimension of consciousness provides a unique, experiential balance
to some challenging material. They punctuate the text with literary quotes,
biographies of key thinkers, and descriptions of their own experiences,
helping readers to reflect on what consciousness is, and what it’s like to be
conscious being. Now.

\end{document}